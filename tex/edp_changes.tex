\documentclass{article}
\usepackage{cite}
\usepackage{amsmath}

\title{Modeling Consumer Response to EDP Changes}
\author{Eric Schulman}
\date{July 2017}

\begin{document}

\maketitle

\section{Introduction}

This proposal explains how I will use the reference price data to model consumer responses to every day price changes. I outline two approaches: a more traditional econometric model; and a dynamic program.

I propose an econometric model similar to the model in Winer's 1986 paper \cite{winer}. In this paper, Winer models the probability consumers buy from a brand as a function of previous quantity sold, consumer price expectations, competitors price and advertising spending. Winer used data directly from the point of sale, so my model is modified to take advantage of the syndicated data.

The second model is a dynamic program that draws on economic theory. It tries to capture the intuition behind a consumers going to the grocery store and making choosing an optimal amount to purchase based on their expectations about future prices. In the model, the consumer updates their expected prices, and choses the optimal amount to consume in each period.

\section{Relevant Literature}

The most relevant areas of research to everyday price changes are research into brand loyalty, reference prices and their relationship with promotions. Reference prices are an expected price. Consumers get a higher utility from purchasing the good below their expected price. Brand loyalty is often represented the statistical odds that consumer will buy the same brand multiple times based on their attitudes. 

Many economists have tried to create econometric models to predict consumers reactions to price changes during promotions due to reference prices and brand loyalty. The most notable model, which I will be modifying, is Winer's econometric model which models the probability consumers buy from a brand as a function of previous quantity sold, consumers price expectations, competitors' price and advertising spending.  \cite{winer}.

Another model worth mentioning is Krishnamurthi and Raj's econometric model which tries to predict the volume of the good purchased. They improve on Winer's model in their 1991 paper using an econometric technique called structural equations \cite{krishnamurthi}. This techniques uses separate econometric equations to model brand loyalty and volume. It produces results that are considered more statistically sound. Also it reflects the intuition that consumers decide on both brand and quantity.



\section{Winer's Model}
My model borrows heavily from Winer's model. Formally, Winer's model describes the probability of purchasing brand $i$ at time $t$ and is given by

$$ {Pr}_{it} = \alpha_0 + \alpha_1 Vol_{it} + \alpha_2 ADV_{it} +  \alpha_3 (\dfrac{P_{it}} {\sum_j P_{jt}} - \hat{P}_{it})+ \alpha_4 \dfrac {P_{it}}{\sum_j P_{jt}} + \epsilon_{it}$$

$${Pr}_{it}$$ is a boolean variable representing whether or not the brand was purchased at time $t$ before it is fit using the regression equation. This is a logistic regression. The data comes from the point of sale. If there are $j$ brands then there $j-1$ more data points are generated for all the brands that were not purchased.

$$Vol_{it}$$ represents the volume at that time.

$$ADV_{it}$$ is a boolean based on the advertising spending.

$$\dfrac{P_{it}} {\sum_j P_{jt}} - \hat{P}_{it}$$ is meant to capture a reaction in the reference price.  In the model, $P_{it}$ is the price charged by brand $i$ at time $t$. $\hat{P}_{it}$ represents the consumers expectation for the price at the current time period period. It is just the estimated ${P}_{it}$ as a function of ${P}_{it-1}$ 

$$ \alpha_4 \dfrac {P_{it}}{\sum_j P_{jt}}$$ is meant to represent the overall pricing environment.

\section{Model Description}

My model emulates Winer's model by including previous quantity sold, consumer price expectations, competitors price. It predicts volume sold by group $i$ in CTA $j$ at time $t$.

$$\Delta Vol_{ijt} = \alpha_0 + \alpha_1 PRICE_{ijt} + \alpha_2 EDP_{ijt} + \alpha_3 DAIRY_{ij} + \alpha_4 flavor_{ij} + \alpha_5 CM_{ij} + \alpha_6 DD_{ij} + \alpha_{7} ID_{ij} + \alpha_{8} PL_{ij} + \alpha_{9} SIZE32_{ij} +\alpha_{10} SIZE64_{ij}  + \alpha_{11} SIZE48_{ij} + \alpha_{13} PRICE_{ijt-1} + \alpha_{13} PRICE_{ijt-2}  + \alpha_{14}{\sum_{ij} VOL_{ijt}}   + \alpha_{15} {\sum_{ij} Price_{ij}}   $$

$$\delta Vol_{ijt}$$ is the weekly change in volume. I chose weekly change because previous volume dominates volume in terms of statistical significant. It makes it easier to identify which models have more explanatory power.

$$PRICE_{ijt}$$ is the price of the group in the CTA

$$EDP_{ijt}$$

$$DAIRY_{ij}$$

$$flavor_{ij}$$

$$CM_{ij}$$

$$DD_{ij}$$

$$ID_{ij} $$

$$PL_{ij} $$

$$SIZE32_{ij}$$

$$SIZE64_{ij} $$

$$SIZE48_{ij}$$

$$PRICE_{ijt-1} $$

$$PRICE_{ijt-2} $$

$$ {\sum_{ij} VOL_{ijt}} $$

$$\sum_{ij} Price_{ij}} $$

\subsection{CTA Dummy Variables}


$$\delta Vol_{ijt} = \alpha_0 + \alpha_1 PRICE_{ijt} + \alpha_2 EDP_{ijt} + \alpha_3 DAIRY_{ij} + \alpha_4 flavor_{ij} + \alpha_5 CM_{ij} + \alpha_6 DD_{ij} + \alpha_{7} ID_{ij} + \alpha_{8} PL_{ij} + \alpha_{9} SIZE32_{ij} +\alpha_{10} SIZE64_{ij}  + \alpha_{11} SIZE48_{ij} + \alpha_{13} PRICE_{ijt-1} + \alpha_{13} PRICE_{ijt-2}  + \alpha_{14}{\sum_{ij} VOL_{ijt}}   + \alpha_{15} {\sum_{ij} Price_{ij}}   + \sum_{i=1}^{31} \alpha_i CTA_i $$

$$ \sum_{i=1}^{31} \alpha_i CTA_i $$

\subsection{Week Dummy Variables}

$$\delta Vol_{ijt} = \alpha_0 + \alpha_1 PRICE_{ijt} + \alpha_2 EDP_{ijt} + \alpha_3 DAIRY_{ij} + \alpha_4 flavor_{ij} + \alpha_5 CM_{ij} + \alpha_6 DD_{ij} + \alpha_{7} ID_{ij} + \alpha_{8} PL_{ij} + \alpha_{9} SIZE32_{ij} +\alpha_{10} SIZE64_{ij}  + \alpha_{11} SIZE48_{ij} + \alpha_{13} PRICE_{ijt-1} + \alpha_{13} PRICE_{ijt-2}  + \alpha_{14}{\sum_{ij} VOL_{ijt}}   + \alpha_{15} {\sum_{ij} Price_{ij}}   + \sum_{i=0}^{156} \alpha_i WEEK_i $$


$$\sum_{i=0}^{156} \alpha_i \WEEK_i $$

\subsection{2 Stage Least Squares}

Winer uses Price like an IV. He does 2 stage least squares. I plan to do the same thing.

\section{Preliminary Results}

\subsection{Basic Model}

\begin{center}
\begin{tabular}{lclc}
\toprule
\textbf{Dep. Variable:}    &        y         & \textbf{  R-squared:         } &      0.263    \\
\textbf{Model:}            &       GLS        & \textbf{  Adj. R-squared:    } &      0.263    \\
\textbf{Method:}           &  Least Squares   & \textbf{  F-statistic:       } &      1491.    \\
\textbf{Date:}             & Thu, 24 Aug 2017 & \textbf{  Prob (F-statistic):} &      0.00     \\
\textbf{Time:}             &     23:05:06     & \textbf{  Log-Likelihood:    } & -9.1863e+05   \\
\textbf{No. Observations:} &       62567      & \textbf{  AIC:               } &  1.837e+06    \\
\textbf{Df Residuals:}     &       62551      & \textbf{  BIC:               } &  1.837e+06    \\
\textbf{Df Model:}         &          15      & \textbf{                     } &               \\
\bottomrule
\end{tabular}
\begin{tabular}{lcccccc}
               & \textbf{coef} & \textbf{std err} & \textbf{t} & \textbf{P$>$$|$t$|$} & \textbf{[0.025} & \textbf{0.975]}  \\
\midrule
\textbf{const} &    1.059e+04  &     3.41e+04     &     0.311  &         0.756        &    -5.62e+04    &     7.74e+04     \\
\textbf{x1}    &   -2.023e+06  &     1.64e+04     &  -123.541  &         0.000        &    -2.06e+06    &    -1.99e+06     \\
\textbf{x2}    &    1.629e+05  &     2.17e+04     &     7.513  &         0.000        &      1.2e+05    &     2.05e+05     \\
\textbf{x3}    &    1689.5385  &     7618.685     &     0.222  &         0.824        &    -1.32e+04    &     1.66e+04     \\
\textbf{x4}    &    4567.2173  &     6859.456     &     0.666  &         0.506        &    -8877.330    &      1.8e+04     \\
\textbf{x5}    &    2.226e+04  &     1.23e+04     &     1.809  &         0.070        &    -1856.427    &     4.64e+04     \\
\textbf{x6}    &    9105.3157  &     1.21e+04     &     0.751  &         0.453        &    -1.47e+04    &     3.29e+04     \\
\textbf{x7}    &    1.393e+04  &     1.28e+04     &     1.092  &         0.275        &    -1.11e+04    &     3.89e+04     \\
\textbf{x8}    &     3.15e+04  &      1.2e+04     &     2.630  &         0.009        &     8023.628    &      5.5e+04     \\
\textbf{x9}    &   -6344.7006  &     1.08e+04     &    -0.588  &         0.557        &    -2.75e+04    &     1.48e+04     \\
\textbf{x10}   &   -2113.0252  &      1.3e+04     &    -0.162  &         0.871        &    -2.76e+04    &     2.34e+04     \\
\textbf{x11}   &    2221.5663  &     1.78e+04     &     0.125  &         0.901        &    -3.26e+04    &      3.7e+04     \\
\textbf{x12}   &    2.313e+06  &     1.82e+04     &   127.049  &         0.000        &     2.28e+06    &     2.35e+06     \\
\textbf{x13}   &   -4.665e+05  &     1.62e+04     &   -28.823  &         0.000        &    -4.98e+05    &    -4.35e+05     \\
\textbf{x14}   &       0.0009  &     7.51e-05     &    12.348  &         0.000        &        0.001    &        0.001     \\
\textbf{x15}   &   -2.289e+05  &      2.3e+04     &    -9.932  &         0.000        &    -2.74e+05    &    -1.84e+05     \\
\bottomrule
\end{tabular}
\begin{tabular}{lclc}
\textbf{Omnibus:}       & 23468.865 & \textbf{  Durbin-Watson:     } &      2.813    \\
\textbf{Prob(Omnibus):} &    0.000  & \textbf{  Jarque-Bera (JB):  } & 16947552.135  \\
\textbf{Skew:}          &   -0.135  & \textbf{  Prob(JB):          } &       0.00    \\
\textbf{Kurtosis:}      &   83.628  & \textbf{  Cond. No.          } &   5.68e+09    \\
\bottomrule
\end{tabular}
%\caption{GLS Regression Results}
\end{center}


\subsection{Adding Dummy Variables}


\subsection{2 Stage Least Squares}

\bibliography{edp_changes}{}
\bibliographystyle{plain}

\end{document}
